\documentclass[10pt,a4paper]{article}
\usepackage[utf8]{inputenc}
\usepackage[english]{babel}
\usepackage[T1]{fontenc}
\usepackage{amsmath}
\usepackage{amsfonts}
\usepackage{amssymb}
\usepackage{makeidx}
\usepackage{graphicx}
\usepackage{fourier}
\usepackage[left=2cm,right=2cm,top=2cm,bottom=2cm]{geometry}
\author{Johannes Scheller, Vincent Noculak, Lukas Powalla}
\title{Computational Physics - Project 2}
\begin{document}
\maketitle
\newpage
\tableofcontents
\newpage
\section{Introduction And Motivation}
In many fields of both mathematics and physics, we often come to the point that we have to solve so-called eigenvalue problems, which are equations of the form $\hat{A}\cdot\hat{v}=\lambda\hat{v}$, where $\hat{A}$ is a matrix of dimension $n\times n$ and $v$ is a vector of dimension $n$. Equations of this kind occur not only in linear algebra, but also in mechanics and quantum mechanics and will also be a major part of this report. In this project, we are going to rewrite the Schrödinger's equation of one and two electrons in a harmonic oscillator potential in the form of an eigenvalue problem and solve it numerically by implementing Jacobi's method, an algorithm that can be used to solve any eigenvalue problem.






\section{Theory}
This theory part deals with the Schrödinger equation for one or two electrons in a tree dimensional harmonic potential. In particular, we want to solve the Schrödinger equation numerically. Therefore, we rewrite the Schrödinger equation as eigenvalue problem and solve the eigenvalue problem with Jacobi's method. 

\subsection{One electron in a harmonic oscillator potential \label{one electron in harm osc}}
We are now looking at one Elektron in a harmonic oscillator potential. 
The time independent Schrödinger equation in general looks like:
\begin{align}
\hat{H} \left|\Psi \right> = E \left|\Psi \right>
\end{align}
We are going to have a closer look at the Schrödinger equation of one electron in a harmonic oscillator. 
You can rewrite the Hamiltonian like it is done in Formula \ref{Schrödinger1}. Here, m is the mass of the particle, $V(r)=\frac{1}{2}k \cdot r^2$ is the potential, $\hat{p_r}=-i \hbar \frac{1}{r} \frac{\partial}{\partial r} r$ is the radial component of the momentum operator, $E=\hbar \omega (2n+l+\frac{2}{3})$ is the Energy and $\hat{L}$ is the Angular momentum operator. 
\begin{align}
\left( \frac{1}{2m} \left[ \hat{p_r}^2 + \frac{\hat{L}^2}{r^2}\right] +V(r) \right) \Psi = E \Psi \label{Schrödinger1}
\end{align}

We can now solve this equation with an product ansatz $\Psi= R(r) \cdot Y(\Theta, \phi )$. The angular-part provides spherical harmonics (as long $V(r)=\frac{1}{2}k \cdot r^2$ is only a function of radius). We are interested in the solution for R(r). After eliminating the angular-dependant part, we receive Formula \ref{Schrödinger2}. (for further information look at "The Physics of Atoms and Quanta -Introduction to Experiments and Theory " written by Herman Haken and Hans Christoph Wolf - Chapter 10: Quantum Mechanics of the Hydrogen Atom)
\begin{align}
-\frac{\hbar^2}{2 m} \left( \frac{1}{r^2} \frac{d}{dr}r^2 \frac{d}{dr}-\frac{l(l+1)}{r^2} \right) R(r)+ V(r) = E R(r) \label{Schrödinger2}
\end{align}
We want to solve this equation numerically. In this project, we are only considering l=0.
In order to have Dirichlet boundary conditions, we substitute $R(r)=\frac{1}{r}\cdot u(r)$. (where r is element of [0,$\infty$)  ) The boundary conditions for the new variable are u(0)=0 and u($\infty$)=0. 

\begin{align}
-\frac{\hbar^2}{2 m}  \frac{d^2}{dr^2} u(r)+ V(r) = E u(r) \label{Schrödinger3}
\end{align}

In order to have more general solution of our equation, we rewrite equation \ref{Schrödinger2} in dimensionless variables. 

we define:
\begin{align}
\alpha&= \left(\frac{\hbar^2}{mk}\right)^{\frac{1}{4}} \\ \label{alpha}
\lambda&=\frac{2m\alpha^2}{\hbar^2}E \\ \label{lambda}
\rho&= \frac{1}{\alpha} r
\end{align}

Finally, we can rewrite the Schrödinger equation in a dimensionless form. (equation \ref{Schrödinger4}) We know already the values for $\lambda$ since we know energy values. (re-substitution using equations \ref{alpha}, \ref{lambda} and the $E_{l=0}$)
\begin{align}
\lambda(n)= 4 \cdot n +3 \\
n=0,1,2,3...
\end{align}

\begin{align}
-\frac{d^2}{d \rho^2} \cdot u(\rho) + \rho^2 u( \rho ) = \lambda u(\rho) \label{Schrödinger4}
\end{align}

We use the standard approximation for the second derivative of u($\rho$) (formula \ref{secondderivative})
\begin{align}
\frac{du(\rho)^2}{d \rho^2}= \frac{u(\rho+h)-2u(\rho)+u(\rho-h)}{h^2} \label{secondderivative}
\end{align}
We also have to define the minimum and maximum of our new variable $\rho$. We choose the minimum of $\rho_{min}=0$. For the maximum, we tried different values. The maximum value of $\rho$ effects the numerical precision of the result. However, this impact will be discussed in chapter \ref{Comparison and results}.

Since we want to solve the given equation numerically, we discretise $\rho$. We choose to have n gridpoints for $\rho$. In this case, $\rho$ and the steplength h are defined as follows:

\begin{align}
h &=\frac{\rho_{max}-\rho_{min}}{n_{step}}\\
\rho_i &= \rho_{min}+i \cdot h \quad, i=0,1,2...
\end{align}
We finally end up with equation \ref{Schrödinger5}. This equation can be written as a Eigenvalue problem. In this case, $\lambda$ is the eigenvalue and the left side of equation \ref{Schrödinger5} can be rewritten as a matrix/vector-product.

\begin{align}
\frac{u(\rho_i+h)-2u(\rho_i)+u(\rho_i-h)}{h^2}  + V_i(\rho_i) \cdot u(\rho_i) &= \lambda u(\rho_i) \label{Schrödinger5}\\
V_i(\rho_i) &= \rho_i^2
\end{align}

\begin{align}
 A \cdot \vec{u}(\rho_i)  = \lambda \cdot \vec{u}(\rho_i) \label{eigenvalueprb}
\end{align}
Where A is defined as:
\begin{equation}
  A:=  \left( \begin{array}{ccccccc} \frac{2}{h^2}+V_1 & -\frac{1}{h^2} & 0   & 0    & \dots  &0     & 0 \\
                                -\frac{1}{h^2} & \frac{2}{h^2}+V_2 & -\frac{1}{h^2} & 0    & \dots  &0     &0 \\
                                0   & -\frac{1}{h^2} & \frac{2}{h^2}+V_3 & -\frac{1}{h^2}  &0       &\dots & 0\\
                                \dots  & \dots & \dots & \dots  &\dots      &\dots & \dots\\
                                0   & \dots & \dots & \dots  &\dots       &\frac{2}{h^2}+V_{n_{\mathrm{step}}-2} & -\frac{1}{h^2}\\
                                0   & \dots & \dots & \dots  &\dots       &-\frac{1}{h^2} & \frac{2}{h^2}+V_{n_{\mathrm{step}}-1}

             \end{array} \right)  \label{matrix A}
\end{equation} 
Now, we have rewritten the Schrödinger equation as a Eigenvalue Problem in Formula \ref{eigenvalueprb} with given Matrix A (Matrix \ref{matrix A}). The given eigenvalue problem (formula \ref{eigenvalueprb}) is now solvable with Jacobis method. (Discussed is capture  \ref{Jacobis method}.)

\subsection{Two interacting electrons in a harmonic oscillator potential}

In this capture, we want to solve two electrons interacting in a harmonic oscillator potential.
We can start with a single electron equation in formula \ref{Schrödinger6}. We did all the previous steps as in capture  \ref{one electron in harm osc}. 

\begin{align}
-\frac{\hbar^2}{2 m}  \frac{d^2}{dr^2} u(r)+ \frac{1}{2}k \cdot r^2 u(r)= E^{(1)} u(r) \label{Schrödinger6}
\end{align}
If we don't consider repulsive coulomb interaction between the two electrons, we get equation \ref{Schrödinger7}. In this case, we have a two Electron Energy and a two electron radial wave function ($u(r_1,r_2)$). 

\begin{align}
\left(-\frac{\hbar^2}{2 m}  \frac{d^2}{dr_1^2} -\frac{\hbar^2}{2 m}  \frac{d^2}{dr_1^2} + \frac{1}{2}k \cdot r_1^2+ \frac{1}{2}k \cdot r_2^2 \right) u(r_1,r_2)= E^{(1,2)} u(r_1,r_2) \label{Schrödinger7}
\end{align}
In order to describe the system with more demonstrative variables, we choose to describe the system with a relative coordinate r and a center-of mass coordinate R instead of using $r_1,r_2$. 
\begin{align}
r &=r_1-r_2 \\
R &= \frac{r_1+r_2}{2}
\end{align}

We can now describe the equation \ref{Schrödinger6} with these new variables. This equation is shown in equation \ref{Schrödinger8}

\begin{align}
\left(-\frac{\hbar^2}{m} \frac{d^2}{dr^2}- \frac{\hbar^2}{4m} \frac{d^2}{dR^2}+\frac{1}{4}k r^2 + k R^2 \right) u (r,R) = E^{(1,2)} \cdot u(r,R) \label{Schrödinger8}
\end{align}

This equation can be solved with a product ansatz for the wave function $u(R,r)= \phi(R) \cdot \xi(r)$. In this case, the Energy $E^{1,2}=E_r+E_R$ is the sum of the relative Energy and the center of mass energy. 
we also want to take the Coulomb interaction between the two electrons. Therefore, we ad the repulsive Coulomb interaction. 
\begin{align}
V_{Coulomb}= \frac{\beta e^2 }{|\vec{r_1}-\vec{r_2}|}= \frac{\beta e^2}{r}
\end{align}
We can separate the equation \ref{Schrödinger8} into a part, which is only dependant on r and one part, which is only dependant on R.  Adding the coulomb term, we derive equation \ref{Schrödinger9} for the r dependence of the Schrödinger equation. 

\begin{align}
\left( - \frac{\hbar^2}{m} \frac{d^2}{dr^2}+ \frac{1}{4} k r^2 + \frac{\beta e^2}{r} \right) \xi(r)= E_r \xi(r) \label{Schrödinger9}
\end{align}
Analogue to chapter \ref{one electron in harm osc}, we rewrite equation \ref{Schrödinger9} and introduce dimensionless variables $\rho=\frac{1}{\alpha}$. 

\begin{align}
\left(-\frac{d^2}{dr^2}+\frac{1}{4}\frac{mk}{\hbar^2} \alpha^4 \rho^2 + \frac{\beta e^2 \alpha m }{\rho \hbar^2} \right) \xi(r) = \frac{m \alpha^2}{\hbar^2} E_r \xi(r) \label{Schrödinger10}
\end{align}
In order to make equation \ref{Schrödinger10} dimensionless, we would like to rewrite it in such a way that we can perform the same steps to solve the equation like in chaper \ref{one electron in harm osc}. 
\begin{align}
\omega_r &= \frac{1}{4} \frac{mk}{\hbar^2} \alpha^4 \\
\alpha &=  \frac{\hbar^2}{m \beta e^2} \\
\lambda &= \frac{m \alpha^2}{\hbar^2} E
\end{align}
The final, dimensionless equation is equation \ref{Schrödinger 11}. 

\begin{align}
\left(-\frac{d^2}{d \rho^2}+ \omega_r^2 \rho^2 + \frac{1}{\rho} \right) \xi(\rho) = \lambda \xi(\rho) \label{Schrödinger 11}
\end{align}

We can solve equation \ref{Schrödinger 11} numerically by approximation of the second derivative in the same way like in chapter \ref{one electron in harm osc}. We just have to reintroduce the potential $V_i$. Afterwards, we can solve the equation \ref{Schrödinger 12} like we solved equation \ref{Schrödinger5}. 

\begin{align}
-\frac{\xi(\rho_i+h)-2u(\rho_i)+\xi(\rho_i-h)}{h^2}+V_i(\rho_i) \cdot \xi(\rho_i) = \lambda \xi(\rho_i) \label{Schrödinger 12}\\
V_i(\rho_i) = \left( \omega_r^2 \rho_i^2 + \frac{1}{\rho_i} \right)
\end{align}


\subsection{Jacobi's method \label{Jacobis method}}

Jacobis method is a iterative method to solve a eigenvalue-problem. 

\begin{align}
A \vec{x} = \vec{b}
\end{align}

It is based on "simple" rotations of the matrices in order to make the biggest Offdiagonal element of the Matrix A to zero. we also have to rotate the eigenvectors while performing the transformation. One can show that this method will always konverge to the right solution if the matrix A is positive definite (all eigenvalues positive) or diagonally dominant. We can split the matrix A up into a upper triangular matrix U, lower triangular matrix L and a diagonal matrix D. ($ A=U + L + D$)
We start with a guess for the unknown. For the n+1 iteration, we get:

\begin{align}
% D \vec{x}^{n+1}+(L+U) \vec{x}^{n}  &=   \vec{b} \\
\vec{x}^{n+1} &= D^{-1} \left( \vec{b}- (L+U) \vec{x}^{n} \right) 
\end{align}













\section{Execution}
\subsection{Implementing the algorithm}
\subsection{Setting and Testing of Parameters}
%Check this section again!
In both cases, whether we deal with only one particle or with two, we have three parameters to be set, resulting in two degrees of freedom that have an effect on the accuracy of our results. The first and most obvious parameter is $n$, the number of grid points we use. Using a higher value of $n$, we gain more precision as the step length decreases, but at the same time, we will end up with a larger matrix that needs more memory (proportional to $n^2$). Most important, the number of similarity transformations needed to calculate the eigenvalues goes like $n^3$, leading to a very long computation time for large matrices. The highest possible value we used for $n$ was $1000$, resulting in more than 45 minutes of computation time!

The second parameter that we can alter is $\rho_{max}$, the maximum value of $\rho$. In theory, this value should be infinite, which is just not possible for this numeric solution. In our case, the higher an eigenvalue is, the more its calculation depends on the choice of $\rho_{max}$. Therefore the challenge was to set this parameter to a value which resulted in stable and consistent results for the first three eigenvalues without being to high, as a higher value would also increase our step length $h$ if we don't change $n$ accordingly.

The last degree of freedom is to set the tolerance for the non-diagonal matrix elements that are supposed to become zero. This value determines implicitly how many similarity transformations are being operated until the non-diagonal elements are considered zero. A smaller value can lead to higher precision in the eigenvalues, but will at the same time increase the computation time again.

We tested different set-ups with different values of $n$, $rho_{max}$ and $epsilon$ with the results shown in table \ref{parameters}. As we wanted a precision of three leading digits for the three lowest eigenvalues, we decided to use the set-up with $n=$, $rho_{max}=$ and $epsilon=$, which seemed to be a good compromise between precision and computation time and led to the desired results.
%Table here!
\subsection{Results}
\section{Comparison and discussion of the results \label{Comparison and results}}
\end{document}