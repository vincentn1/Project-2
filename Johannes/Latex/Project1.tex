\documentclass[10pt,a4paper]{article}
\usepackage[utf8]{inputenc}
\usepackage[english]{babel}
\usepackage[T1]{fontenc}
\usepackage{amsmath}
\usepackage{amsfonts}
\usepackage{amssymb}
\usepackage{makeidx}
\usepackage{graphicx}
\usepackage{fourier}
\usepackage[left=2cm,right=2cm,top=2cm,bottom=2cm]{geometry}
\author{Johannes Scheller, Vincent Noculak, Lukas Powalla}
\title{Computational Physics - Project 2}
\begin{document}
\maketitle
\newpage
\tableofcontents
\newpage
\section{Introduction And Motivation}
In many fields of both mathematics and physics, we often come to the point that we have to solve so-called eigenvalue problems, which are equations of the form $\hat{A}\cdot\hat{v}=\lambda\hat{v}$, where $\hat{A}$ is a matrix of dimension $n\times n$ and $v$ is a vector of dimension $n$. Equations of this kind occur not only in linear algebra, but also in mechanics and quantum mechanics and will also be a major part of this report. In this project, we are going to rewrite the Schrödinger's equation of one and two electrons in a harmonic oscillator potential in the form of an eigenvalue problem and solve it numerically by implementing Jacobi's method, an algorithm that can be used to solve any eigenvalue problem.

\section{Theory}
\subsection{Rewriting Schrödinger's equation as eigenvalue problem}
\subsubsection{One electron in a harmonic oscillator potential}
\subsubsection{Two interacting electrons in a harmonic oscialltor potential}
\subsection{Jacobi's method}
\section{Execution}
\subsection{Implementing the algorithm}
\subsection{Results}
\section{Comparison and discussion of the results}
\end{document}